\documentclass[11pt]{article}
\usepackage{aahomework}
\usepackage{mathtools}
\usepackage{subcaption}
\usepackage{epstopdf}
\usepackage{float}
\usepackage{xcolor}
\usepackage{parskip}
%\tikzstyle{blk}=[circle,inner sep=0pt,minimum size =4pt,draw,fill=black,line width=0.8pt]
%\tikzstyle{blanknode}=[circle,inner sep=3pt,minimum size =8pt,draw,line width=0.8pt]
%\tikzstyle{blk}=[circle,inner sep=0pt,minimum size =4pt,draw,fill=black,line width=0.8pt]
%
%\geometry{letterpaper, textwidth=17cm, textheight=22cm}

%\usetikzlibrary{arrows}
%\usetikzlibrary{plotmarks}

\newcommand{\ques}{\paragraph{Question:}}
\newcommand{\keyphrase}{\textbf}
\newcommand{\boxit}[2]{\textcolor{#1}{\boxed{\textcolor{black}{#2}}}}

\title{Lecture 3}
\author{A. Agarwal}
\date{December 7, 2012}


\begin{document}
%\maketitle

\section*{Recall}

\begin{enumerate}
\item{
Row-echelon form
}
\item{
Reduced row echelon form (rref)
}
\item{
Row operations
\begin{enumerate}
\item{
Multiply a row by a scalar: $\boxit{black}{cR_i}$
}
\item{
Swap rows: $\boxit{black}{R_i \circlearrowright R_j}$
}
\item{
Add a multiple of one row to another: $\boxit{black}{R_i + cR_j}$
}
\end{enumerate}
}
\end{enumerate}

It is important to note that a matrix can have many row-echelon forms -- row echelon form is not unique. It is only the \emph{reduced} row echelon form that is unique.

Let us take a look at an example. Consider the system
\begin{align*}
\left.
\begin{aligned}
x + 3y &= 4
\\
5x - 4y &= -7
\end{aligned}
\right\}
\end{align*}
First, we represent this system as an augmented matrix. Our goal then is to perform row operations to achieve one of two forms
\begin{align*}
\left[\begin{matrix}
1 & 3 & 4
\\
5 & -4 & -7
\end{matrix}\right]
\longrightarrow
\underbrace{
\left[\begin{matrix}
1 & 0 & *
\\
0 & 1 & *
\end{matrix}\right]
\text{ or }
\left[\begin{matrix}
1 & 0 & *
\\
0 & 0 & 1
\end{matrix}\right]
}_{\text{rref.}}
\end{align*}
After we achieve one of these forms, we can determine much of the information that we want to know about the system.

\begin{align*}
\left[\begin{matrix}
1 & 3 & 4
\\[.4em]
5 & -4 & -7
\end{matrix}\right]
\
\xrightarrow[]{-5R_1 + R_2}
\
\left[\begin{matrix}
1 & 3 & 4
\\[.4em]
0 & -19 & -27
\end{matrix}\right]
\
\xrightarrow[]{-\frac{1}{19} R_2}
\
\left[\begin{matrix}
1 & 3 & 4
\\[.4em]
0 & 1 & \frac{27}{19}
\end{matrix}\right]
\end{align*}

Both of these matrices are in echelon form. If we want to find the reduced row echelon form of the matrix, then we still need to eliminate non-zero entries in the columns above leading non-zero entries:
\begin{align*}
\left[\begin{matrix}
1 & 3 & 4
\\[.4em]
0 & 1 & \frac{27}{19}
\end{matrix}\right]
\
\xrightarrow[]{-3R_2 + R_1}
\
\underbrace{
\left[\begin{matrix}
1 & 0 & \frac{-5}{19}
\\[.4em]
0 & 1 & \frac{27}{19}
\end{matrix}\right]
}_{\text{rref.}}
\end{align*}
We have now obtained the reduced row echelon form of the augmented matrix. Our goal was to solve the original system. We can see how the reduced row echelon form helps us find the solution by converting it back into a system:
\begin{align*}
\left[\begin{matrix}
1 & 0 & \frac{-5}{19}
\\[.4em]
0 & 1 & \frac{27}{19}
\end{matrix}\right]
\longrightarrow
\left\{
\begin{aligned}
x & & &= \frac{-5}{19}
\\[.4em]
&&y &= \frac{27}{19}
\end{aligned}
\right.
\end{align*}
We see that the solution to the system comes very easily from the reduced row echelon form.

The leading non-zero entries in the echelon and reduced row echelon form matrices were very important in this process. We can now introduce some new terminology.

\begin{itemize}
\item{
A \keyphrase{pivot entry} is the leading non-zero entry in a matrix in echelon form. Non-pivot entries are simply those entries that are not pivot entries.
}
\item{
A \keyphrase{pivot column} or basic column is the column containing a pivot entry.
}
\item{
A \keyphrase{free column} is a column that does not contain pivot entries.
}
\end{itemize}

Looking back at the previous example,
\begin{align*}
\left[\begin{matrix}
\boxit{red}{1} & 3 & 4
\\[.4em]
0 & \boxit{red}{1} & \frac{27}{19}
\end{matrix}\right]
\end{align*}
We find the matrix has two pivot entries (red), and so this matrix has two pivot columns and one free column.


These ideas have meaning in the sense of variables as well.
\begin{itemize}
\item{
A \keyphrase{pivot variable} is a variable of the system corresponding to a pivot column.
}
\item{
A \keyphrase{free variable} is a a variable corresponding to a free column.
}
\end{itemize}
One obvious but important fact is that the sum of the pivot and non-pivot variables equals the total number of variables in the system.

If we consider the previous example, we find there are two pivot variables and no free variables:

\begin{align*}
\left[\begin{matrix}
1 & 0 & \frac{-5}{19}
\\[.4em]
0 & 1 & \frac{27}{19}
\end{matrix}\right]
\end{align*}

With this terminology in mind, let us address the question of whether we can predict the number of solutions a system will have.

Suppose we have some system that we reduce to this form.
\begin{align*}
\text{System}
\longrightarrow
\left[
\begin{matrix}
1 & 0 & 0 & 2
\\
0 & 0 & 0 & 3
\\
0 & 0 & 1 & 1
\end{matrix}
\right]
\end{align*}
Examining this matrix, we find it has three pivot entries:
\begin{align*}
\left[\begin{matrix}
\boxit{red}{1} & 0 & 0 & 2
\\
0 & 0 & 0 & \boxit{red}{3}
\\
0 & 0 & \boxit{red}{1} & 1
\end{matrix}\right]
\end{align*}
This then tells us that
\begin{itemize}
\item{
There are two pivot entries, which are both $1$.
}
\item{
There are two pivot columns: the first and third.
}
\item{
There are two free columns: the second and fourth.
}
\end{itemize}
To address which variables are free and which are pivots, we need to associate a variable with each column. Suppose the variables of the first, second, and third columns are, respectively, $x,y,z$.
\begin{itemize}
\item{
There are two pivot variables: $x$ and $y$.
}
\item{
There is one free variable: $y$.
}
\end{itemize}

In presenting this problem, note that we were not actually given a matrix in echelon form. The reason the matrix does not qualify as echelon is due to the fact that there are non-zero entries in the third row that are not to the right of the leading non-zero entry in the second row.

To fix this, we need to perform additional row operations.
\begin{align*}
\left[
\begin{matrix}
1 & 0 & 0 & 2
\\
0 & 0 & 0 & 3
\\
0 & 0 & 1 & 1
\end{matrix}
\right]
\xrightarrow[]{R_2 \circlearrowright R_3}
\left[
\begin{matrix}
1 & 0 & 0 & 2
\\
0 & 0 & 1 & 1
\\
0 & 0 & 0 & 3
\end{matrix}
\right]
\end{align*}
The matrix is now in echelon form.

Let us reassess this matrix. First, it is in echelon form but not rref due to the fact that the leading non-zero entry in the third row is a $3$. The pivot entries are now different:
\begin{align*}
\left[
\begin{matrix}
\boxit{red}{1} & 0 & 0 & 2
\\
0 & 0 & \boxit{red}{1} & 1
\\
0 & 0 & 0 & \boxit{red}{3}
\end{matrix}
\right]
\end{align*}
\begin{itemize}
\item{
Pivots: $1,1,3$
}
\item{
Pivot columns: $1,3,4$
}
\item{
Free columns: $2$
}
\item{
Pivot variables: $x,z$
}
\item{
Free variables: $y$
}
\end{itemize}

Let us convert this matrix into a system of equations:

\begin{align*}
\left[
\begin{matrix}
1 & 0 & 0 & 2
\\
0 & 0 & 1 & 1
\\
0 & 0 & 0 & 3
\end{matrix}
\right]
\longrightarrow
\left\{
\begin{aligned}
x && && &= 2
\\
&& && z &= 1
\\
&& && 0 &= 3
\end{aligned}
\right.
\end{align*}
Observe that this tells us that $x=2$, $z=1$ and $y$ can be anything. However, observe that it also says that $0=3$.

Because the system of equations says that $0=3$, the system can have no solution. This is thus an example of an \emph{inconsistent} system, as we have discussed previously.

This suggests that we might be able to tell when this will occur by looking at the echelon form matrix. Thinking about it, it is apparent that whenever a pivot is in the constant vector column, the resulting system will have such an inconsistent equation.


Succinctly,

\begin{minipage}[t]{.95\textwidth}
Suppose we have a system, represented by an augmented matrix
\begin{align*}
\left[
\begin{matrix}
A & \vec{b}
\end{matrix}
\right]
\end{align*}
which we reduce it to
\begin{align*}
\left[
\begin{matrix}
R & \vec{c}
\end{matrix}
\right]
\end{align*}
where $R$ is echelon or rref. of $A$ and this reduction results in $\vec{b} \to \vec{c}$.

If the right side column of $\left[
\begin{matrix}
R & \vec{c}
\end{matrix}
\right]$ has a pivot, then the system is inconsistent.
\end{minipage}


\ques Is the converse also true? That is, that if a system is inconsistent, that a pivot must be present in the constant column?

The answer to this will come later.

It seems that pivot and free variables determine the nature of the system. Let us consider another example system:
\begin{align*}
\left[
\begin{matrix}
1 & 2 & 3 & 4 & 0
\\
0 & 0 & 2 & 1 & 2
\\
0 & 0 & 0 & 7 & 5
\end{matrix}
\right]
\end{align*}
We'll assign variables to the columns: $x,y,z,w$ for columns $1$ through $4$, respectively.
\begin{enumerate}
\item{
Pivot columns: 1,3,4
}
\item{
Free columns: 2,5
}
\item{
Pivot variables: $x,z,w$
}
\item{
Free variables: $y$
}
\end{enumerate}

The first question we probably want to answer is: is the system consistent or inconsistent? Let us convert the augmented matrix back into a system:
\begin{align*}
\begin{aligned}
x + 2y + 3z + 4w &= 0
\\
2z + w &= 2
\\
7w &= 5
\end{aligned}
\end{align*}
From here, it is apparent that we can use the third equation to obtain a value for $w$. We can then plug this value into the second equation to obtain a value for $z$. Plugging these two into the first equation, we might think we can then obtain the value of $x$. But, remember that $y$ is a free variable. After isolating the values of $w$ and $z$, we should then be able to pick any value of $y$ and be able to find the value of $x$ that satisfies the values of $w,z,$ and $y$. This tells us that the system is consistent.

Let us actually solve the system. The last line gives us that $w = 5 / 7$. Then from the second equation,
\begin{align*}
2z + w &= 2
\\
z &= \frac{2-w}{2} = \frac{2 - \frac{5}{7}}{2} = \frac{9}{14}
\end{align*}
Next, we can use these two values in the first equation to obtain
\begin{align*}
x + 2y + 3z + 4w &= 0
\\
x &= -2y - 3z -4w
\\
&= -2y - \frac{27}{14} - \frac{20}{7}
\\
&= -2y - \frac{47}{14}
\end{align*}
We haven't selected any value for $y$ yet, even though since it is a free variable we should be able to pick any value we want for it. For example, we can select $y=0$ and then we simply have that $x = -47/14$. If we pick $y= -2$, we obtain $y=-2-(47/14) = -75/14$.

We find that for every $y \in \bbR$ we pick, there is a $x$ corresponding to it. Hence the system is consistent. Additionally, we have found much more than just that the system is consistent, we have in fact found that it will have infinitely many solutions.


Having a free variable makes it \emph{possible} for the system to have infinitely many solutions. We just need to be careful that the system has any solutions at all; \emph{there could be an inconsistent system that still has free variables}. In that case, there would still be no solutions to the system.

Otherwise, it is the case that for a consistent system, having a free variable means it will have infinitely many solutions.

It still remains to be seen what determines whether there will be a unique solution to a system.

Before we move on, let us discuss how to write the solution to a system. We have found the solution to a system already, but there is a more convenient way to present such a solution, called the \keyphrase{vector form} of a solution.


Looking at the previous example, we will first write the variables of the system in a column vector
\begin{align*}
\begin{bmatrix}
x
\\[.4em]
y
\\[.4em]
z
\\[.4em]
w
\end{bmatrix}
&=
\end{align*}

We isolated specific values for $w$ and $z$. Since $y$ is free we will simply write $y=y$, and we will write the expression for $x$ that depends on $y$. Since $y$ is free it is good practice to write the choice of $y$; in this case, $y$ can be any real number ($y \in \bbR$).
\begin{align*}
\begin{bmatrix}
x
\\[.4em]
y
\\[.4em]
z
\\[.4em]
w
\end{bmatrix}
&=
\begin{bmatrix}
-2y - \frac{47}{14}
\\[.4em]
y
\\[.4em]
\frac{9}{14}
\\[.4em]
\frac{5}{7}
\end{bmatrix}
\qquad , y \in \bbR
\end{align*}
We will always present the \emph{solution set} in this form.

Next, let us try writing the solution in a slightly different manner:
\begin{align*}
\begin{bmatrix}
x
\\[.4em]
y
\\[.4em]
z
\\[.4em]
w
\end{bmatrix}
&=
\begin{bmatrix}
\frac{-47}{14}
\\[.4em]
0
\\[.4em]
\frac{9}{14}
\\[.4em]
\frac{5}{7}
\end{bmatrix}
+
y
\begin{bmatrix}
-2
\\[.4em]
1
\\[.4em]
0
\\[.4em]
0
\end{bmatrix}
\qquad ,y \in \bbR
\end{align*}
There are some benefits of splitting the solution in this way. First, this form indicates that the solution, in four dimensions, is actually a line. We can interpret this by considering a lower-dimensional case.

\begin{figure}[H]
\centering
Particular point
Vector along line
\end{figure}

In two dimensions, a line is determined by any particular point on the line, and the vector that lies along the line. Back in the current example, the first vector represents the particular point, whereas the second is the vector lying along the line.
\begin{align*}
\begin{bmatrix}
x
\\[.4em]
y
\\[.4em]
z
\\[.4em]
w
\end{bmatrix}
&=
\underbrace{
\begin{bmatrix}
\frac{-47}{14}
\\[.4em]
0
\\[.4em]
\frac{9}{14}
\\[.4em]
\frac{5}{7}
\end{bmatrix}
}_{\text{Particular point on the line}}
+
y
\overbrace{
\begin{bmatrix}
-2
\\[.4em]
1
\\[.4em]
0
\\[.4em]
0
\end{bmatrix}
}^{\text{Vector along the line}}
\qquad ,y \in \bbR
\end{align*}

This may be familiar to those who have taken differential equations. The particular point is analogous to the particular solution, whereas the vector part of the solution is the homogeneous solution.

\end{document}
